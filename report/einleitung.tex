\chapter{Introduction}

\section{Project Context and Objectives}

Context of this project is ongoing research for high performance network interface cards at the Integrated Electronics Systems Labs by Boris Traskov, also supervisor of this project.

In our environment an increasing amount of data is created by every kind of device. Furthermore, the amount of transferred data over networks (i.e. the internet) increases dramatically. This generates a demand for high speed network interfaces, on the one hand and high performance, low cost full stack network implementations on the other hand. 

Service providers and back-end nodes need to be able to transfer large amounts of data to multiple receivers concurrently, requiring high speed network interfaces. To circumvent - potentially unnecessary - \gls{cpu} utilization, much of the work can be "offloaded" to a network interface, providing a higher level of abstraction to the transfer of data, than current state of the art systems provide.

In a future scenario with all devices being connected to each other, often described as the "internet of things", an increasing demand for simple to implement network interfaces, not requiring the presence of an operating system or high performance processors, will be created.

%The ability to provide network interfaces at low cost, becomes especially relevant in the future "`internet of things"', describing a connection of almost all devices among each other.

Objectives for the whole project entitled "Design of an Accelerated Event-based Server" is to setup a hardware system on a \textit{Xilinx} \gls{fpga} utilizing \gls{ip} cores, running \textit{nginx} (an event-based web server) on top. Furthermore benchmarks have to be conducted, to prove the strength of such a system. Ideally proving that the system is capable of utilizing the full Gigabit Ethernet interface without breaking down on attempted \gls{dos} attacks.

Desired goal for this project seminar was to lay the necessary foundations, so that outstanding implementation work and extensive measurements can be completed in a subsequent bachelor thesis.

\section{Hardware Platform and Tool Set}

All development, measurements and tests were done on the \textit{Xilinx XUPV5-LX110T} evaluation board.

The \textit{Xilinx XUPV5-LX110T} evaluation board is a modified version of the \textit{ML505} board for universities \cite{xupv5manual}. The difference between the two boards is a larger \gls{fpga} chip on the \textit{XUPV5-LX110T}, containing the \textit{Virtex-5 XC5VLX110T} with 17,280 slices\footnote{Basic logic unit of an \gls{fpga} \cite{fpga_ni}} and four in hardware implemented Ethernet \gls{mac} cores, whereas the \textit{ML505} board contains the \textit{XC5VLX50} with only 7,200 slices and no Ethernet \gls{mac} cores implemented in hardware.

Building the hardware design for the \gls{fpga} was started using the \textit{\gls{xps}} and continued with the \textit{Xilinx ISE Project Navigator}, for a finer control over the parameters of used tools and better logging and reporting capabilities. All used \textit{Xilinx} tools are part of the \textit{Xilinx ISE Design Suite} in version 14.1. The synthesis tool (xst) of this version of the tool suite features parallel synthesis speeding up the development process heavily.

Software development was done using the \textit{Xilinx Software Development Kit (SDK)}.

\chapter{System Requirements}

Requirements for the designed system must be collected with two key points in mind: the desired outcome of the project and the chosen/available hardware system and tool set.

Objective of the project is to run \textit{nginx}, an event-based web server. \textit{nginx} can not be executed directly on the processor, because it requires the presence of an an \gls{os} providing a file system and taking responsibility of process and thread management. 

\textit{nginx} supports a number of free (FreeBSD, Solaris, Linux) and proprietary (AIX, HP-UX) unix-based operating systems and Microsoft Windows.\footnote{see \url{http://nginx.org/en/\#tested_os_and_platforms}}

Microsoft Windows is currently only available for Intel's \textit{x86} and AMD's \textit{AMD64} (also known as \textit{x86-64}) architectures. The constraint on the available \textit{Xilinx XC5VLX110T} \gls{fpga} demands an easy-to-implement, low cost microprocessor system. But the mentioned processor architectures do not count towards these categories. Therefore the choices are limited to free unix-based operating systems.

Due to the by far largest number of supported processor architectures and wide distribution, the choice was made to go with Linux as operating system for the project.

Linux demands an \gls{mmu} in virtual mode and two memory protection zones.\footnote{Linux Kernel can be configured for processors without \gls{mmu}, but this is not recommended.} \gls{mmu}s enable an \gls{os} "to exercise a high degree of
management and control over its address space and the address space it allocates to processes". \cite{linuxPrimer}[sec. 2.3.5]

Furthermore the presence of two timers and an interrupt controller is required.\footnote{see \url{http://wiki.xilinx.com/microblaze-linux\#toc4}, as of 09/07/2012}


\chapter{Implementation}

\section{The Hardware System}

\subsection{FPGAs}

A \gls{fpga} consists of a large number of slices. In the \textit{Xilinx Virtex-5} \gls{fpga} family every slice contains four \gls{lut} and two flip-flops. The \gls{lut}s define how the flip-flops and slices are connected to each other and can be programmed with a so called \textit{bitstream}. Therefore a \gls{fpga} can be programmed with any hardware layout, containing logical gates and flip-flops, without the expensive and time-consuming process of producing electronic chips out of silicon wafers.

Additionally to these slices a \gls{fpga} may contain already implemented hardware cores, which can be used by a system implemented using the slices of the \gls{fpga}. These hardware cores can not be "overridden" and might contain any kind of system or device, like access controllers for external hardware or full processors.

\subsection{Configuration}

An initial \gls{soc} design can be created using \gls{xps}. This tool contains a wizard for creating new designs, by selecting the target board and favored \gls{ip} cores. 

By default this wizard does not provide an option for the \textit{Xilinx XUPV5} board. For adding this board to the wizard the respective \gls{bsp} needs to be downloaded from the \textit{Xilinx} website\footnote{\url{http://www.xilinx.com/univ/xupv5-lx110t-bsb.htm}}. After extracting the downloaded zip file the directory named "\texttt{Xilinx\_XUPV5\_LX110T}" needs to be copied to the respective directory ("\texttt{ISE\_DS/EDK/board/Xilinx/boards}") inside of the \textit{Xilinx} tools installation.

If no modifications can be made to the \textit{Xilinx} tools installation to add the \textit{Xilinx XUPV5} board \gls{bsp}, the \textit{ML505} board can be selected in the wizard instead. After the project is created, the target device needs to be set to "\texttt{virtex5, xc5vlx100t, ff1136, -1}" in the project options.

All properties chosen in the setup wizard or later on in \gls{xps} are stored in the \textit{\gls{mhs}} file in the root directory of the \gls{xps} project. All pin associations and other (timing) constraints are located in the \gls{ucf}. These should be replaced by the constraints specified in the \gls{ucf} file referenced in the report appendix.

\subsection{Architecture}

In this project a \gls{soc} was build using predefined \gls{ip} cores for the processor and additional system components.

An overview of the complete hardware architecture of the system described in the next sections is included in the appendix of this report (\ref{sec:hw_arch}).

\subsubsection{Processor}
\label{subsubsec:microblaze}

The used \textit{XC5VLX110T} \gls{fpga} does not have a build-in hard core processor like available in the \textit{Xilinx XC5VFX} series. Therefore a so called soft-core microprocessor needs to be implemented on the \gls{fpga}. There exists a wide range of soft microprocessor cores.\footnote{see \url{http://en.wikipedia.org/wiki/Soft_microprocessor}} To avoid additional licensing fees and because of the tight time schedule of the project, a single core \textit{MicroBlaze} processor was chosen. The \textit{MicroBlaze} processor is a proprietary processor, developed by \textit{Xilinx} for their \gls{fpga} families and supported by the Xilinx hard- and software development kits. Its design follows the Harvard architecture with separate data and instruction memory.

Running Linux kernel requires the presence of a \gls{mmu}. This can be activated by the following settings in the project's \gls{mhs} file for the \textit{MicroBlaze} core. For using a \gls{mmu} it is also required to have optimization for area disabled (default value):

\begin{verbatim}
 PARAMETER C_USE_MMU = 3
 PARAMETER C_MMU_ZONES = 2
 
 PARAMETER C_AREA_OPTIMIZED = 0
\end{verbatim}

To improve the system performance it is recommended to enable instruction and data caches (16 KB), barrel shifter, multiplier (64 bit) and the hardware division module. Therefore the \textit{MicroBlaze} \gls{ip} core settings in the \gls{mhs} file should be adjusted to meet the following values:

\begin{verbatim}
 PARAMETER C_USE_BARREL = 1
 PARAMETER C_USE_HW_MUL = 2
 PARAMETER C_USE_DIV = 1

 PARAMETER C_USE_ICACHE = 1
 PARAMETER C_USE_DCACHE = 1

 PARAMETER C_CACHE_BYTE_SIZE = 16384
 PARAMETER C_ICACHE_ALWAYS_USED = 1
 PARAMETER C_ICACHE_LINE_LEN = 8
 PARAMETER C_ICACHE_STREAMS = 1
 PARAMETER C_ICACHE_VICTIMS = 8
 PARAMETER C_DCACHE_BYTE_SIZE = 16384
 PARAMETER C_DCACHE_ALWAYS_USED = 1
\end{verbatim}

Automatic processor version recognition by the Linux Kernel (optional) can be activated by enabling the \gls{pvr}:

\begin{verbatim}
 PARAMETER C_PVR = 2
\end{verbatim}

\subsubsection{Bus System}

For connecting the \textit{MicroBlaze} processor to other peripherals on the chip a bus system needs to be selected. The \gls{xps} wizard provides an option for choosing between two interconnect types: \gls{axi} and \gls{plb}.

\gls{axi} is part of the \gls{amba}, designed by \textit{ARM} and introduced for \gls{fpga}s by \textit{Xilinx} starting with the \textit{Virtex-6} \gls{fpga} family generation.

Prior to the \textit{Virtex-6} \gls{fpga} family, only \gls{plb} invented by \textit{IBM} as part of the \textit{CoreConnect} bus system was available. \textit{Virtex-6} \gls{fpga}s support both bus systems, but the used \textit{Xilinx XC5VLX110T} \gls{fpga} is part of the \textit{Virtex-5} family, therefore \gls{plb} needs to be selected as interconnect type. \cite{axi_interconnect}[p. 1, facts table]

\subsubsection{Memory}

The \textit{XUPV5-LX110T} board contains a 9 MB ZBT synchronous SRAM, a single-rank unregistered 256 MB DDR2 SODIMM and multiple flash memory chips. The DDR2 memory fits the requirements for the desired system best. Both regarding read and write speed, as well as the provided amount of memory.

To connect the memory to the processor a memory controller needs to be added to the system.  \textit{Xilinx} provides the \textit{Multi-Port Memory Controller} \gls{ip} core (MPMC) -- select-able in the \gls{xps} project wizard -- for this purpose.

The \textit{MPMC} IP core can be connected to \gls{plb} and supports \textit{Xilinx CacheLink (XCL)} structures and \textit{Soft Direct Memory Access (SDMA)} for LocalLink interfaces, required for fast processing of the network interface core. \cite{mpmc}

The designed hardware system uses \texttt{0x50000000} as memory base address in this project. The address can be changed arbitrarily, but needs to be adjusted accordingly in the Linux kernel configuration (see "Required Configuration Settings",  \ref{subsubsec:req_config}).

\subsubsection{Network Interface}

The \textit{Xilinx XC5VLX110T} \gls{fpga} has four \textit{Tri-Mode Ethernet Media Access Controllers}, designed to the IEEE 802.3-2002 specification, operating at 10, 100, and 1,000 Mb/s. \cite{virtex5}[p. 4, table 1] To use these hard core controllers an \texttt{xps\_ll\_temac} soft IP core can be added to the \gls{soc}, acting as a wrapper for the hard core to integrate it into the system.

For a simple and easy-to-implement interconnection between the \gls{phy} and the various \gls{mac} sub-layers, \gls{mii} was invented. \gls{mii} was designed originally for \gls{phy}s with data rates of 10 Mb/s and 100 Mb/s. \gls{gmii} is an backwards compatible extension to \gls{mii} supporting data rates of up to 1,000 Mb/s. \textit{RMII/RGMII} is an interface with reduced data path width (by half). This is accomplished by clocking data on rising and falling edges of the clock. \cite{rgmii}[p. 3]

For the project \gls{gmii} was selected as physical interface type, because support for Gigabit Ethernet was desired, but there was no need for a reduced data path width. Therefore the parameter \texttt{C\_PHY\_TYPE} needs to be set to \texttt{1} on the \gls{ip} core configuration.

Despite setting the \texttt{C\_PHY\_TYPE} parameter, the jumpers \texttt{J22} and \texttt{J23} on the \textit{Xilinx XUPV5} board need to be set to positions \texttt{1-2} to enable \gls{gmii} as physical interface type.

Depending on the (temporarily) selected \gls{fpga} during the \gls{xps} project wizard, it might be required to change the location of \textit{IDELAYCTRL elements} (set by the \texttt{C\_IDELAYCTRL\_LOC} parameter) to "\texttt{IDELAYCTRL\_X0Y4-IDELAYCTRL\_X1Y5}".

Through the parameters \texttt{C\_TEMAC0\_TXCSUM} and \texttt{C\_TEMAC0\_RXCSUM} offloading checksum calculation to hardware circuits can be enabled, to improve performance. Though there were no measurements done to prove any performance boost during the course of this project, due to lack of time.

\subsubsection{Other Peripherals}

Besides the previously discussed \gls{ip} cores, the following additional standard \gls{ip} cores were used in the system, not requiring any special explanation or configuration.

\begin{itemize}
  \setlength{\itemsep}{1pt}
  \setlength{\parskip}{2pt}
  \setlength{\parsep}{0pt}
\item 2x \texttt{lmb\_v10} -- Local Memory Bus for instruction and data memory.
\item \texttt{bram\_block} -- Internal Random Access Memory block on the \gls{fpga}.
\item 2x \texttt{lmb\_bram\_if\_cntlr} -- Local Memory Bus (LMB) Block RAM (BRAM) Interface Controller for instruction and data memory.
\item \texttt{mdm} -- MicroBlaze Debug Module providing register access and execution control for debugging.
\item \texttt{xps\_intc} -- Interrupt controller.
\item 4x \texttt{xps\_gpio} -- General Purpose I/O controllers for DIP switches, LEDs and push buttons.
\item \texttt{xps\_iic} -- Inter-Integrated Circuit (IIC) controller for access to external EEPROM chip.
\item \texttt{xps\_timer} -- Timer block providing two independent timers.
\item \texttt{xps\_uartlite} -- Universal Asynchronous Receiver Transmitter (UART) controller.
\item \texttt{clock\_generator} -- Generator for system clock (configuration discussed in \ref{sec:clocks}).
\item \texttt{proc\_sys\_reset} -- Processor System Reset Module integrating an external reset pin into the system.
\end{itemize}

The configuration settings and constraints for all \gls{ip} cores as used in this project are provided in the appendix of this report.

\subsection{Clocks}
\label{sec:clocks}

\subsubsection{Clock Generation}

Clocks for the system can be generated using a \textit{clock\_generator} \gls{ip} core. This \gls{ip} core can generate up to 16 different clocks from one input clock signal. On the \textit{Xilinx XUPV5} board an external oscillator providing a 100 MHz clock can be connected to the clock generator. \cite{ug347}[p. 20]

\subsubsection{Timing Considerations}

After synthesis the minimum clock period is reported as \textit{7.566 ns} (i.e. a maximum clock frequency of \textit{132.170 MHz}). Therefore it could be assumed that a 125.00 MHz system clock frequency as select-able in the \gls{xps} wizard should work well, but the Xilinx place and route tool (\textit{par}) is not able to meet all timing constraints for this frequency with enabled \gls{mmu} and instruction cache. The reason for unmet constraints are high delays on data paths in the decode pipeline stage, demanding a clock period of about \textit{9.12 ns}.

Considering these constraints on the clock period, the system clock was set to 100 MHz. This clock is used by the processor and the local bus.

\subsubsection{Clock Signals for Peripherals}

The memory controller (MPMC) demands atleast three different clocks: a base clock, a clock with half the frequency of the base clock and a clock signal with the same frequency as the base clock, but shifted by 90°. If the frequency of the base clock is not 200 MHz a fourth clock with this frequency is required. All clocks need to be controller by the same \gls{pll}.

Therefore the clock pins of the memory controller were connected to the following signals:

\begin{verbatim}
 PORT MPMC_Clk0 = clk_200_0000MHzPLL0
 PORT MPMC_Clk0_DIV2 = clk_100_0000MHzPLL0
 PORT MPMC_Clk90 = clk_200_0000MHz90PLL0
 PORT MPMC_Clk_200MHz = clk_200_0000MHzPLL0
\end{verbatim}

The Ethernet \gls{mac} \gls{ip} core requires a clock signal with exactly 125 MHz at the \texttt{GTX\_CLK} port for operating GMII. This is defined in the specifications of GMII \cite{ieee802_3}[sec. 35.2.2.1]. For \gls{dma} a clock signal with a frequency identical to the local bus clock is required. The \texttt{REFCLK} needs to be connected to a 200 MHz clock, according to the respective manual of the \gls{ip} core \cite{xps_ll_temac}[p. 11, table 3].

\begin{verbatim}
 PORT GTX_CLK_0 = clk_125_0000MHz
 PORT REFCLK = clk_200_0000MHzPLL0
 PORT LlinkTemac0_CLK = clk_100_0000MHzPLL0
\end{verbatim}

The ratio between the MPMC base clock and the clock signal for LocalLink \gls{dma} access of the Ethernet \gls{mac} IP core ($200:100 = 2$) needs to be set to the parameter \texttt{C\_SDMA2\_PI2LL\_CLK\_RATIO} of the MPMC core.

The clock generator is configured to generate a buffered 200 MHz clock on \texttt{CLKOUT3} and a buffered 200 MHz clock with a phase shift of 90° on \texttt{CLKOUT2}. All utilizing the same \gls{pll} unit (PLL0).

\texttt{CLKOUT1} is configured to provide a buffered 125 MHz clock. The buffered 100 MHz system clock is set to port \texttt{CLKOUT0}:

\begin{verbatim}
 PORT CLKOUT0 = clk_100_0000MHzPLL0
 PORT CLKOUT1 = clk_125_0000MHz
 PORT CLKOUT2 = clk_200_0000MHz90PLL0
 PORT CLKOUT3 = clk_200_0000MHzPLL0
\end{verbatim}


\subsection{Endianness}

\begin{quote}
 "Endianness describes how multi-byte data is represented by a computer system and is dictated by the CPU architecture of the system." \cite{intel_endiannness}[p. 5]
\end{quote}

Architectures utilizing the little endian concept store the least significant byte (LSB) at the lowest address, in big endian architectures the most significant byte (MSB) is stored at the highest address. \cite{intel_endiannness}[p. 6]

Linux can be build for little, as well as for big endian systems. Only confinement is that the used toolchain (compiler, etc. - see \ref{subsec:sdk}) needs to support the endianness of the architecture.

The \textit{MicroBlaze} processor has the parameter \texttt{C\_ENDIANNESS} to specify the endianness of the processor. But although the \textit{MicroBlaze Processor Reference Guide} states that "the \texttt{C\_ENDIANNESS} parameter is automatically set to little endian when using AXI4, and to big endian when using PLB, but can be overridden by the user" \cite{mb_ref}[p. 52], this parameter must not be changed for \textit{Virtex-5} \gls{fpga}s. This is reasoned in the disability of the peripheral cores connected via \gls{plb}, to handle data other than in big endian byte order. The \gls{axi} bus circumvents this problem by swapping bytes.\footnote{\url{http://forums.xilinx.com/t5/EDK-and-Platform-Studio/Memory-Test-fails-for-8-and-16-bit/m-p/253922/highlight/true\#M23973} (in-official statement by a Xilinx employee)}

Therefore big endian must be selected for the system architecture of this project.

\subsection{Generating the Hardware System}

As discussed in the sub section about \gls{fpga}s, a \textit{bitstream} needs to be generated from the \gls{mhs} and constraint files.

Therefore the hardware design must be implemented by triggering the "Implement Design" process listed in the \textit{Xilinx ISE Project Navigator}. In a second step the \textit{bitstream} file (\texttt{<project-name>.bit}) can be generated from the implemented design, by triggering the "Generate Programming File" process.

Additionally the hardware design can be exported to the \textit{Xilinx SDK}. This is required for generating small C projects for testing the hardware and single \gls{ip} cores and the generation of further, intermediate configuration files (see "\textit{Device Tree}", \ref{subsubsec:device_tree}).


\section{The Software}

\subsection{Linux Kernel}

Running Linux on a \textit{MicroBlaze} processor is unfortunately rarely documented and not wide spread. Mainly two projects exist on this topic: \textit{PetaLinux} and \textit{BlueCat Linux}. Unfortunately both were turned into commercial solutions and although the patched Linux source code is publicly available, it is hardly usable without the proprietary toolkit, documentation and support, sold by the companies driving these projects.

When talking about Linux, it needs to be distinguished between the Linux kernel as the actual \gls{os} and Linux distributions (like \textit{Ubuntu}, \textit{Debian}, \textit{openSUSE}, \textit{Fedora} and many more), containing additional libraries, applications and configuration. The Linux kernel in itself is the heart of a Linux installation. There exists one major development line (called "mainline"), being maintained by a group around the founder of Linux, Linus Torvalds. But there exist also also copies of the source code (called "fork") with incorporated custom changes.

Originally support for \textit{MicroBlaze} soft-core processors in the Linux kernel was developed by \textit{Xilinx} and \textit{PetaLogix} in an own fork of Linux kernel, but with version 2.6.30, released on the 9th of June 2009, their additions were merged into the mainline Linux kernel and the \textit{MicroBlaze} architecture is officially supported by Linux since this date.\footnote{\url{http://www.petalogix.com/news/microblaze-architecture-merged-into-kernel.org}} The still maintained and actively used source code repositories of \textit{Xilinx}\footnote{\url{http://git.xilinx.com/?p=linux-xlnx.git}} and \textit{PetaLogix}\footnote{\url{http://developer.petalogix.com/git/gitweb.cgi?p=linux-2.6-microblaze.git}} contain some further additions to device drivers and unreleased changes.

For this project the Linux kernel source code from the \textit{Xilinx} repository was taken. All development, additions and evaluation was done on top of commit \texttt{ed63a4d76efadcb68f5776e4244766ffea226cc4} (June, 18th 2012)\footnote{\url{http://git.xilinx.com/?p=linux-2.6-xlnx.git;a=commit;h=ed63a4d76efadcb68f5776e4244766ffea226cc4}} as part of version \texttt{3.3.0-14.2-build1}.

To enable correct recognition of the latest \textit{MicroBlaze} processor versions with enabled \gls{pvr} a patch included in the appendix of this report (\ref{subsec:pvr_patch}) needs to be applied to the Linux kernel sources. The patch was extracted from the \textit{PetaLogix} Linux kernel fork.

\subsection{Development Environment and Cross Compilation}
\label{subsec:sdk}

To build the Linux kernel a set of tools (called toolchain) compiling source code files and linking binary output in an executable file is required. This toolchain needs to meet two criteria: the architecture/\gls{os} it was build to run on top and the target architecture for which images are compiled and linked. Building an image with a different target architecture, than the architecture of the executing system, is called \textit{cross compilation}. Because it is not practically possible to build the Linux kernel on a \textit{MicroBlaze} processor, without an existing and running \gls{os} on the \textit{MicroBlaze} system, a toolchain for cross compiling Linux kernel source from an existing, easily-usable system for a \textit{MicroBlaze} system, is needed. \textit{Xilinx} provides toolchains, based on the widely used \gls{gcc} and \textit{binutils}, for cross compiling from Linux x86 and x86-64 architectures to \textit{MicrobBlaze} systems as target architecture. For compiling the latest Linux kernel sources, version two of the toolchain containing \gls{gcc} in version 4.1.2 is required. This version of the toolchain itself requires a 64-bit Linux \gls{os} and can be downloaded from the \textit{Xilinx} source code repositories.\footnote{\url{http://git.xilinx.com/?p=xldk/microblaze_v2.0.git}}

During this project a virtual Linux machine with \textit{Ubuntu 12.04 64-bit} was used for all software development.

To enable cross compilation of the Linux kernel, two environment variables need to be set. The \texttt{CROSS\_COMPILE} variable specifies a prefix for all build tools. This is required, because on cross compilation there is usually more than one compiler available on the executing system. Furthermore the path to the utilized build tools needs to be added to the \texttt{PATH} variable.

The \texttt{ARCH} variable enables the inclusion of code specific to the target architecture. This code files reside in \texttt{/arch/<ARCH>/} inside the Linux kernel source folder. 

The environment variables can be set as parameter on build execution or for the complete shell session with the following commands:

\begin{verbatim}
export PATH=/<path>/<to>/microblaze-unknown-linux-gnu/bin:$PATH
export CROSS_COMPILE=microblaze-unknown-linux-gnu-
\end{verbatim}

A build process is triggered by calling the \texttt{make} executable inside of the Linux kernel source folder. Here an example for triggering a new build of the Linux kernel for the \textit{MicroBlaze} architecture, utilizing two processor cores:

\begin{verbatim}
make -j 2 ARCH=microblaze simpleImage.xupv5
\end{verbatim}

\texttt{simpleImage} specifies the image format, \texttt{xupv5} is an additional configuration option explained in the \textit{Device Tree} section (\ref{subsubsec:device_tree}).

\subsubsection{The Linux Configuration System}

Linux kernel consists of many optional sub-parts for target architectures, device drivers, special features, etc. Which of these parts are compiled and linked into the Linux kernel binary image needs to be configured in the \texttt{.config} file in the Linux kernel root directory. This file is "the configuration blueprint for building a Linux kernel image" \cite{linuxPrimer}[sec. 4.3.1] containing all (required) settings.

Although possible, it is not recommended to edit the \texttt{.config} file manually, because of many non-trivial dependencies among different configuration settings. Instead one of the available editors (\textit{menuconfig}, \textit{xconfig} and \textit{gconfig}) should be used. Because of its search capabilities inside the configuration settings, \textit{xconfig} was used as configuration editor during this project.\footnote{\textit{xconfig} requires the \textit{QT} library which can be installed through the package \texttt{qt4-dev-tools} on Debian and Ubuntu distributions.}

Before starting a new project all existing configuration and temporary output files should be cleaned up using the commands "\texttt{make ARCH=microblaze clean}" and "\texttt{make ARCH=microblaze mrproper}".

In the Linux kernel fork maintained by \textit{Xilinx} a base configuration for the \textit{ML507} board is included. It is recommended to use this configuration as base line instead of starting from scratch. The configuration can be loaded using the following command:

\begin{verbatim}
make ARCH=microblaze xilinx_mmu_defconfig
\end{verbatim}

From there on the configuration can be viewed and edited using a configuration editor with the following command:
\begin{verbatim}
make ARCH=microblaze xconfig
\end{verbatim}

\subsubsection{Required Configuration Settings}
\label{subsubsec:req_config}

As discussed in \ref{subsubsec:microblaze}, the \textit{MicroBlaze} processor can be configured with different feature sets (multiplier, barrel shifter, etc.). Therefore the \gls{gcc} compiler needs to be parameterized for matching the provided features of the target system \cite{mb_linux}[sec. "Kernel Configuration Details"]. These need to be set in the \texttt{XILINX\_MICROBLAZE0\_*} settings inside of the \texttt{.config} file.

It is important to check that the \texttt{KERNEL\_BASE\_ADDR} setting is in line with the configured base address of the main memory (\texttt{C\_MPMC\_BASEADDR}) in the \textit{\gls{mhs}} file.

For inclusion of the driver for the \texttt{xps\_ll\_temac} \textit{\gls{ip} core}, the configuration option \texttt{XILINX\_LL\_TEMAC} needs to be enabled. Support for \textit{UART} device drivers can be enabled by the settings \texttt{SERIAL\_UARTLITE}, \texttt{SERIAL\_UARTLITE\_CONSOLE}, \texttt{SERIAL\_CORE} and \texttt{SERIAL\_CORE\_CONSOLE}, but this is already included in the default \textit{ML507} configuration settings provided by \textit{Xilinx}.

When there is no \textit{boot loader} triggering Linux kernel start, command-line options for the boot process need to be compiled into the Linux kernel image. Therefore the configuration setting \texttt{CMDLINE} exists. The command-line options for this project were set as follows:

%The setting \texttt{CMDLINE\_FORCE} indicates whether the specified command-line options, can be overridden by a \textit{boot loader} (\texttt{no}) or not (\texttt{yes}). 

\begin{verbatim}
CONFIG_CMDLINE = "console=ttyUL0 root=/dev/ram rw"
\end{verbatim}

\texttt{ttyUL0} sets the first serial device (index 0) of type \textit{uartlite} (\texttt{UL}) as default console and loads the RAM image with read and write access (\texttt{rw}) as root file system (see "The File System" \ref{subsec:fs}).

Additionally some features like \textit{IPv6} support and various security algorithms were enabled in the Linux kernel configuration. The complete \texttt{.config} file with all settings is referenced in the report appendix.

\subsubsection{Device Tree}
\label{subsubsec:device_tree}

To avoid compiling all information about the target hardware system into assembler code, an abstraction called \textit{Device Tree} was invented. \textit{Device Tree} is the successor of \textit{IBM's} \textit{Open Firmware (OF)} and the flattened device tree, containing information about the hardware system like available feature sets and bus addresses. \cite{device_tree} It is accessed by the Linux kernel during the boot process for configuring itself and on lookup of hardware information. The \textit{Device Tree Source (dts)} file is compiled by the \textit{Device Tree Compiler (dtc)} during the Linux kernel build process to an \textit{Device Tree Blob (dtb)} and linked into the final Linux kernel image.

The \textit{dts} file must reside inside of the \texttt{arch/microblaze/boot/dts} directory (e.g. "\texttt{arch/microblaze/boot/dts/xupv.dts}") and can be specified on starting the Linux kernel build process (e.g. "\texttt{make ARCH=microblaze simpleImage.xupv5}").

A \textit{dts} file can be generated using the \textit{Device Tree Generator}, a \gls{tcl} script reading a system specification generated by \gls{xps} (see "SDK Export"). The latest version (1.3) of the \textit{Device Tree Generator}, maintained by Michal Simek, is referenced in the appendix of this report (\ref{subsec:dts-generator}).

Apart from the generated device information, the \textit{dts} file also contains options which must be set manually. These include an \texttt{aliases} node for mapping device instances to standardized interface names (\texttt{ethernet0} and \texttt{serial0}) and command-line options, overriding\footnote{By setting the kernel configuration option \texttt{CMDLINE\_FORCE} to "\texttt{y}", the command-line options specified in the \texttt{.config} file take precedence.} the ones specified in the Linux kernel configuration file. Additionally the \texttt{stdout} device bus address can be set:

\begin{verbatim}
aliases {
    ethernet0 = &Hard_Ethernet_MAC;
    serial0 = &RS232_Uart_1;
} ;
chosen {
    bootargs = "console=ttyUL0 ip=192.168.3.10 root=/dev/ram";
    linux,stdout-path = "/plb@0/serial@84020000";
} ;
\end{verbatim}

Besides the options already explained in the "\textit{Required Configuration Settings}" section (\ref{subsubsec:req_config}), this configuration assigns the logical IPv4 address \texttt{192.168.3.10} to the network interface of the board.

The \textit{Device Tree Generator} can be setup as own \gls{bsp} inside of the \textit{Xilinx SDK}. A documentation with all required steps for setting it up can be found on the respective site hosted by \textit{Xilinx}.\footnote{\url{http://wiki.xilinx.com/device-tree-generator}}


\subsection{The File System}
\label{subsec:fs}

Although possible, it makes little sense to use the Linux kernel without a file system. Therefore one of the last steps in the boot process is the initialization of a \textit{root file system (rootfs)}. \cite{linuxPrimer}[sec. 6.1] On modern desktop and server systems this \textit{rootfs} is usually just a bare minimum file system, containing all necessary files to boot the Linux kernel, mounting a file system located on a hard drive or flash memory right after booting up. But the used system in this project has no hard drive attached, neither is one required for the purpose of the project, at the moment. Therefore we will stick with the initial \textit{rootfs} as the main file system.

A simple way to provide an initial \textit{rootfs} is the \textit{Initial RAM Disk (initrd)}. This is a file system packed in a \textit{cpio} archive and linked into the Linux kernel image. It is unpacked completely into the main memory during kernel boot process. \textit{Xilinx} provides two packed file system archives for a \textit{MicroBlaze} system within their Linux kernel repository: \texttt{initramfs\_minimal.cpio.gz} and \texttt{initramfs\_complete.cpio.gz}. Both contain all files and structures sufficient for this project. The archives can be linked into the Linux kernel image by setting the configuration option \texttt{INITRAMFS\_SOURCE} to one of the file names, respectively.

\textit{cpio} archives can be unpacked using the following command. This command should be executed as privileged root user to allow the creation of node points, used in the file system.

\begin{verbatim}
gzip -dc /<path>/<to>/initramfs_complete.cpio.gz | sudo cpio -i
\end{verbatim}

After all changes were made to the files representing the file system, it can be packed into a \textit{cpio} archive using the bash script supplied in the appendix of this report ("\texttt{pack-fs.sh}", \ref{subsec:pack-fs}). The paths to the Linux kernel sources and file system root need to be adjusted to meet the current environment.


\subsection{Deployment and Debugging}

The generated \textit{bitstream} of the hardware system representing the \gls{soc}, needs to be programmed into the \gls{fpga} on every power-on. This is required, because the configuration of the \gls{fpga} being set by the \textit{bitstream} is volatile. Programming the \textit{bitstream} is straight forward and can be accomplished using the \textit{Xilinx iMPACT} tool, which is part of the \textit{Xilinx ISE Design Suite}.

When the Linux kernel build is finished, the resulting \textit{\gls{elf}} file can be found at \texttt{arch/microblaze/boot/simpleImage.<dts-name>}. This file can be loaded into the \gls{fpga} using \textit{\gls{xmd}}, which is also part of the \textit{Xilinx ISE Design Suite}. The steps for loading an \gls{elf} file onto a \gls{fpga} with \textit{MicroBlaze} processor using \gls{xmd} are as follows:

\begin{enumerate}
  \setlength{\itemsep}{1pt}
  \setlength{\parskip}{2pt}
  \setlength{\parsep}{0pt}
\item Connecting to the \textit{MicroBlaze} processor:\\ \texttt{xmd> connect mb mdm}
\item Stopping the \textit{MicroBlaze} processor:\\ \texttt{xmd> stop}
\item Downloading the \gls{elf} file:\\ \texttt{xmd> dow simpleImage.<dts-name>}
\item Starting the \textit{MicroBlaze} processor:\\ \texttt{xmd> con}
\end{enumerate}

After committing these steps the Linux kernel boot log should be written to the connected serial interface. However, when there are problems in the early boot process, the console output might not have been initialized yet. If this is the case, the content written to the console output can be read and decoded from memory manually. It is stored in the global variable "\texttt{\_\_log\_buf}".

Reading the \texttt{\_\_log\_buf} variable from memory and decoding the binary representation of the content can be accomplished using the \gls{tcl} script included in the report appendix (\textit{syslog.tcl}, \ref{subsec:syslog}). It is a ported version for \textit{MicroBlaze} processors of the script supplied with the \textit{Xilinx Application Note XAPP1137}\footnote{\url{http://www.xilinx.com/support/documentation/application\_notes/xapp1137.pdf}}. 

One thing that needs to be adjusted according to the Linux kernel image is the start address of the \texttt{\_\_log\_buf} variable (at line 6, "\texttt{set bufaddr <addr>}"). This address can be determined by executing the command "\texttt{xmd> mrd\_var \_\_log\_buf}" inside of \gls{xmd}. The command returns the word stored at the first address of the variable and the address of the variable itself: "\texttt{\_\_log\_buf(<address>)   <data>}". After saving the \gls{tcl} script to a file it can be loaded into a \gls{xmd} session using the command "\texttt{xmd> source syslog.tcl}". The script writes all the content of the \texttt{\_\_log\_buf} variable to the file "\texttt{syslog\_1.txt}" and can be executed using the command "\texttt{xmd> syslog}".

\subsection{Testing the System}

Full functionality of the system, including the network interface, can be tested by connecting a computer to the Ethernet port of the board and starting a small web server on the \texttt{MicroBlaze} processor.

This can be done executing the following commands on the console connected to the system:

\begin{verbatim}
mkdir /var/www
echo "Hello FPGA!" > /var/www/index.html
httpd -p 80 -h /var/www
\end{verbatim}

The network interface of the connected computer needs to be configured to be part of the subnet \texttt{192.168.3.0/24}, except the address \texttt{192.168.3.10} which is assigned to the system on the board.

The test site returning the content "\textit{Hello FPGA!}" can be accessed from the connected computer by visiting the site \url{http://192.168.3.10/} in a web browser.

\chapter{Conclusion}

\section{Project Review and Evaluation}

I chose this project, because it promised to cover many topics of personal interest, including network systems, performance oriented research and optimization, web technology and System on chip design. These expectations were fulfilled, but came with a steep learning curve.

After some weeks with administrative problems concerning setup of the environment and required tools, the project could get started. The following two to three weeks were filled with learning the utilized tools and understanding the overall process of designing an \gls{ip} core based \gls{soc} on a \gls{fpga}. When being familiar with the general concepts and topics covered by this project, I started to focus on the actual project goal, assembled a time schedule and defined an, from my point of view at that time, doable intermediate goal for the project seminar. But as it turned out during the course of the project, it was hardly possible to stick to the schedule and the planned outcome of the project seminar. This was mainly reasoned in lots of problems occurring at points, which initially seemed to be straight forward implementation tasks, but in fact contained many non-trivial pitfalls. 

The major difficulties solved during this project seminar were that there exists some documentation and reference projects (partly) covering this project, on the one hand, but these could not be ported easily to the tools and hardware family used in this project. This was especially a problem, because \textit{Xilinx} makes little effort to keep new versions of their tools and included \gls{ip} cores compatible with older hardware families. Discovering and keeping this circumstance in mind, was one of the key points in successfully implementing the described outcome of the project seminar.

All in all working through this project seminar was a hard but valuable experience, especially learning about all the interfaces and dependencies between the different levels in the stack of a computer system.

\section{Outlook}

The project seminar covered only the first part of the project goal. So there is naturally some work to be done to finish the project. This covers especially everything around incorporating custom user space applications (like \textit{nginx}) and an accurate measurement and evaluation of performance parameters of the designed system.

Besides these outstanding topics, improvements to already implemented parts of the system could be made, too. To name a few, the baud rate of the serial \gls{ip} core could be increased to enable a more responsive console interface for accessing the system. An additional, non-volatile file system could be added to the system to circumvent the need for programming a complete image onto the \gls{fpga} on every persisted change to the file system. This could be the inclusion of a \textit{Network File System (NFS)} or a file system stored on a flash drive.

When the complete system is implemented, including the event-based server, as stated in the project title the focus of the project work needs to be shifted more towards measuring the performance parameters of the system. As part of this work an "\textit{TCP Offload Engine}", designed in a parallel project could be incorporated into the system. Purpose of this offload engine is to fork processing load resulting from incoming and outgoing network traffic from the \gls{cpu} to a designated hardware core. It is expected that this will speed up the overall performance of the system, allowing either the usage of less hardware resources or a higher network throughput.

The next steps in a subsequent project seminar or bachelor thesis would be to get  cross compilation of user space applications working, incorporate \textit{nginx} into the system and conduct the mentioned performance measurements.