\documentclass[parskip,
			oneside,
			%longdoc,
			10pt,
			noheadingspace,
			accentcolor=tud1d,
			bigchapter,
			%draft,
			colorback]{tudreport}

%% Spracheinstellungen
\usepackage[ngerman]{babel}
\usepackage[utf8]{inputenc}
\usepackage[T1]{fontenc} 
\usepackage{microtype} % optischer Randausgleich bei pdflatex mit Zeichendehnung

%% Grafikeinstellungen
\usepackage{float} % u.a. genaue Plazierung von Gleitobjekten mit H
\usepackage{wrapfig}
\addto\captionsngerman{%
  \renewcommand{\figurename}{Abb.}%
}

%% Tabelleneinstellungen
\usepackage{booktabs}
\usepackage{multirow}
\usepackage{longtable}
\usepackage{tabularx}

%% Mathematik
\usepackage{amsmath}
\usepackage{nicefrac}
\usepackage{icomma}

%% sonstige Einstellungen
\usepackage{paralist}% erweiterte Listenumgebung (z.B. compactitem)
\usepackage{textcomp} % verschiedene Symbole
\usepackage{hyperref}
\renewcommand\plparsep{1ex}
\usepackage{enumerate}

\usepackage[toc]{glossaries}
\usepackage{makeidx}
\makeindex

\usepackage[fixlanguage]{babelbib}
\selectbiblanguage{german}


\title{Design of an Accelerated Event-based Server}
\subtitle{Project Seminar Report by Peter Schuster}
\subsubtitle{Advisor: Dipl.-Ing. Boris Traskov\\
			 Start: 02.05.2012 \textbar\ Abgabe: 31.12.2011\\
			 Institut of Computer Engineering\hfill\textbar\hfill Integrierte Elektronische Systeme Lab \hfill\textbar\hfill Prof.\,Dr.-Ing.\, Klaus Hofmann}
\setinstitutionlogo{ies-logo.pdf}
\institution{Institut of Computer Engineering\\Integrierte Elektronische Systeme Lab\\Prof. Dr.-Ing. Klaus Hofmann}
%\settitlepicture{images/spartan}
\begin{document}

%% Titel %%%%%%%%%%%%%%%%%%%%%%%%%%%%%%%%%%%%%%%%%%%%%%%%%%%%%%%%%%%%%%%%%%
\maketitle
\cleardoublepage

%% Vorgeplnkel %%%%%%%%%%%%%%%%%%%%%%%%%%%%%%%%%%%%%%%%%%%%%%%%%%%%%%%%%%%%%%
\pagestyle{empty}
%\pagenumbering{none}
\pagenumbering{arabic}

\tableofcontents

\newpage
\phantomsection %notwendig, damit Link nicht unterhalb der Überschrift zeigt
\addcontentsline{toc}{chapter}{List of figures}
\listoffigures


% include glossary

%
% Glossary
%

%\newglossaryentry{FPGA}{Field Programmable Gate Array}
\newacronym{fpga}{FPGA}{Field Programmable Gate Array}
\makeglossaries

%% Hauptteil %%%%%%%%%%%%%%%%%%%%%%%%%%%%%%%%%%%%%%%%%%%%%%%%%%%%%%%%%%%%%%%
\pagestyle{headings}


\chapter{Einleitung}

Das Projektseminar steht im Kontext der Forschung an High-Performance Netzwerkkarten zur Übertragung von Daten mit bis zu 40 GB/s. TCP Offload Engine

\section{Ziel}

Ziel des Projektseminars ist es einen \ac{FPGA} zu programmieren um über eine Ethernet-Schnittstelle Daten zwischen dem Evaluationsboard und einem weiteren Gerät auszutauschen und Lasttests durchzuführen.

\section{Testumgebung}

Für die Versuche, Messungen und Tests wird das Evaluationsboard ML505 von XILINX verwendet. Die Programmierung des \ac{FPGA} erfolgt über das \textit{Xilinx ISE Studio}, bzw. \textit{Xilinx SDK}.



\chapter{Implementierung}

\chapter{Evaluation}

\chapter{Ausblick}

%% Anhang %%%%%%%%%%%%%%%%%%%%%%%%%%%%%%%%%%%%%%%%%%%%%%%%%%%%%%%%%%%%%%%%
\appendix

\clearpage
\printglossaries

\clearpage
% \nocite{*}
%\newpage %neue Seite, damit Link auf Seitenanfang zeigt
%\phantomsection %notwendig, damit Link nicht unterhalb der Überschrift zeigt
\addcontentsline{toc}{chapter}{Literature} %Eintrag im Inhaltsverzeichnis
\bibliographystyle{plain}
\bibliography{literaturverzeichnis}


\end{document}
