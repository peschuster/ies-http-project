\chapter{System Requirements}

Requirements for the designed system must be collected with two key points in mind: the desired outcome of the project and the chosen/available hardware system and tool set.

Objective of the project is to run \textit{nginx}, an event-based web server. \textit{nginx} can not be executed directly on the processor, because it requires the presence of an an \gls{os} providing a file system and taking responsibility of process and thread management. 

\textit{nginx} supports a number of free (FreeBSD, Solaris, Linux) and proprietary (AIX, HP-UX) unix-based operating systems, as well as \textit{Microsoft Windows}.\footnote{see \url{http://nginx.org/en/\#tested_os_and_platforms}}

\textit{Microsoft Windows} is currently only available for Intel's \textit{x86} and AMD's \textit{AMD64} (also known as \textit{x86-64}) architectures. The constraint on the available \textit{Xilinx XC5VLX110T} \gls{fpga} demands an easy-to-implement, low cost microprocessor system. But the mentioned processor architectures do not count towards these categories. Therefore the choices are limited to free unix-based operating systems.

Due to the by far largest number of supported processor architectures and wide distribution, the choice was made to go with Linux as operating system for the project.

Linux demands a \gls{mmu} in virtual mode and two memory protection zones.\footnote{Linux kernel can be configured for processors without \gls{mmu}, but this is not recommended.} \gls{mmu}s enable an \gls{os} "to exercise a high degree of
management and control over its address space and the address space it allocates to processes" \cite{linuxPrimer}[sec. 2.3.5]. Linux kernel furthermore requires the presence of two timers and an interrupt controller.\footnote{see \url{http://wiki.xilinx.com/microblaze-linux\#toc4}, as of 09/07/2012}

The used web server software (\textit{nginx}) requires the availability of an event module in the \gls{os}. Linux kernel contains a number of event modules, all being supported by \textit{nginx}. The preferred and most effective one is \textit{epoll}.\footnote{\url{http://wiki.nginx.org/NginxOptimizations} (as of 12/2012)}