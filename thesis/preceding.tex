\chapter{Preceding Work}

Preceding to this bachelor thesis, a project seminar with the same title was taken. During this project seminar the first part of the project was implemented. The outcome of this project seminar is described in the following chapter. A detailed description on how this was accomplished can be found in the respecting project report \cite{projectseminar}.

\section{The Hardware System}

\subsection{Architecture}

In this project a \gls{soc} was build on top of a \gls{fpga} using predefined \gls{ip} cores for the processor and additional system components.

An overview of the complete hardware architecture of the system described in the next sections is included in the appendix of this report (\ref{sec:hw_arch}).

\subsubsection{Processor}
\label{subsubsec:microblaze}

The used \textit{XC5VLX110T} \gls{fpga} does not have a build-in hard core processor, therefore a single core \textit{MicroBlaze} soft core processor was chosen. The \textit{MicroBlaze} processor is a proprietary processor, developed by \textit{Xilinx} for their \gls{fpga} families and supported by the Xilinx hard- and software development kits. Its design follows the Harvard architecture with separate data and instruction memory.

Running Linux kernel requires the presence of a \gls{mmu}. To improve the system performance instruction and data caches (16 KB), barrel shifter, multiplier (64 bit) and the hardware division modules were enabled.

\subsubsection{Bus System}

For connecting the \textit{MicroBlaze} processor to other peripherals on the chip the \gls{plb}, invented by \textit{IBM} as part of the \textit{CoreConnect} bus system, was selected.

Prior to the \textit{Virtex-6} \gls{fpga} family, only this bus system was available. \textit{Virtex-6} \gls{fpga}s support also the \gls{axi} system, which is part of the \gls{amba}, designed by \textit{ARM}, but the used \textit{Xilinx XC5VLX110T} \gls{fpga} is part of the \textit{Virtex-5} family, therefore \gls{plb} needed to be selected as interconnect type. \cite{axi_interconnect}[p. 1, facts table]

\subsubsection{Memory}

\textit{XUPV5-LX110T} boards contain a single-rank unregistered 256 MB DDR2 SODIMM, which is connected to the processor via a memory controller. This memory controller is implemented in the \textit{Multi-Port Memory Controller} (MPMC) \gls{ip} core. The memory base address was set to \texttt{0x50000000}.

\subsubsection{Network Interface}

The \textit{Xilinx XC5VLX110T} \gls{fpga} has four \textit{Tri-Mode Ethernet Media Access Controllers}, designed to the IEEE 802.3-2002 specification, operating at 10, 100, and 1,000 Mb/s. \cite{virtex5}[p. 4, table 1] To use these hard core controllers an \texttt{xps\_ll\_temac} soft IP core was added to the \gls{soc}, acting as a wrapper for the hard core to integrate it into the system.


\gls{gmii} as a backwards compatible extension to \gls{mii} supporting data rates of up to 1,000 Mb/s was selected as physical interface type, because support for Gigabit Ethernet was desired, but there was no need for a reduced data path width. Therefore the jumpers \texttt{J22} and \texttt{J23} on the \textit{Xilinx XUPV5} board need to be set to positions \texttt{1-2} to enable \gls{gmii} as physical interface type.

Usage of an integrated checksum calculation circuit is enabled on the system, using the parameters \texttt{C\_TEMAC0\_TXCSUM} and \texttt{C\_TEMAC0\_RXCSUM}.

\subsection{Clocks}
\label{sec:clocks}

Clocks for the system are generated using a \textit{clock\_generator} \gls{ip} core, with an external oscillator providing a 100 MHz clock. \cite{ug347}[p. 20]

Due to high delays on data paths in the decode pipeline stage, a clock period of at least \textit{9.12 ns} is required, resulting in a system clock frequency for the processor and local bus of 100 MHz.

The memory controller (MPMC) is driven by base clock of 200 MHz, a clock with half the frequency of the base clock (100 MHs) and a 200 MHz clock signal, shifted by 90°. All these clock signals are controlled by the same \gls{pll} used by the system clock signal.

The \texttt{GTX\_CLK} port of the Ethernet \gls{mac} \gls{ip} core is driven by a clock signal with exactly 125 MHz for operating \textit{GMII} (defined by the specifications of GMII \cite{ieee802_3}[sec. 35.2.2.1]). For \gls{dma} a clock signal with a frequency identical to the local bus clock is required. The \texttt{REFCLK} was connected to a 200 MHz clock, according to the respective manual of the \gls{ip} core \cite{xps_ll_temac}[p. 11, table 3].

\subsection{Endianness}

\begin{quote}
 "Endianness describes how multi-byte data is represented by a computer system and is dictated by the CPU architecture of the system." \cite{intel_endiannness}[p. 5]
\end{quote}

Architectures utilizing the little endian concept store the least significant byte (LSB) at the lowest address, in big endian architectures the most significant byte (MSB) is stored at the highest address. \cite{intel_endiannness}[p. 6]

Linux can be build for little, as well as for big endian systems. Only confinement is that the used toolchain (compiler, etc. - see \ref{subsec:sdk}) needs to support the endianness of the architecture.

The \textit{MicroBlaze} processor has the parameter \texttt{C\_ENDIANNESS} to specify the endianness of the processor. But although the \textit{MicroBlaze Processor Reference Guide} states that "the \texttt{C\_ENDIANNESS} parameter is automatically set to little endian when using AXI4, and to big endian when using PLB, but can be overridden by the user" \cite{mb_ref}[p. 52], this parameter must not be changed for \textit{Virtex-5} \gls{fpga}s. This is reasoned in the disability of the peripheral cores connected via \gls{plb}, to handle data other than in big endian byte order. The \gls{axi} bus circumvents this problem by swapping bytes.\footnote{\url{http://forums.xilinx.com/t5/EDK-and-Platform-Studio/Memory-Test-fails-for-8-and-16-bit/m-p/253922/highlight/true\#M23973} (in-official statement by a Xilinx employee)}

Therefore big endian was selected for the system architecture of this project.

\section{The Software}

\subsection{Linux Kernel}

Chosen \gls{os} of the system is the \textit{Linux kernel}. This is the pure core Linux \gls{os}, in comparison to enriched Linux distributions (like \textit{Ubuntu}, \textit{Debian}, \textit{openSUSE}, \textit{Fedora} and many more), which contain additional libraries, applications and configuration. The Linux kernel version used in this project is based on release 3.3.0 and contains some further additions and bug fixes by \textit{Xilinx}.

To enable correct recognition of the latest \textit{MicroBlaze} processor versions with enabled \gls{pvr} a patch included in the appendix of this report (\ref{subsec:pvr_patch}) needs to be applied to the Linux kernel sources. The patch was extracted from the \textit{PetaLogix} Linux kernel fork.

To build the Linux kernel a set of tools (called toolchain) compiling source code files and linking binary output in an executable file is required. \textit{Xilinx} provides toolchains, based on the widely used \gls{gcc} and \textit{binutils}, for cross compiling from Linux x86 and x86-64 architectures to \textit{MicrobBlaze} systems as target architecture.

\subsubsection{Configuration}

\begin{wrapfigure}{r}{.65\textwidth}
%	\begin{figure}[H]
	\centering
	\includegraphics[width=.6\textwidth]{linux-config-build.png}
	\caption{Components of a Linux kernel build.}
%	\end{figure}
\end{wrapfigure}

Linux kernel consists of many optional sub-parts for target architectures, device drivers, special features, etc. Which of these parts are compiled and linked into the Linux kernel binary image needs to be configured in the \texttt{.config} file in the Linux kernel root directory. This file is "the configuration blueprint for building a Linux kernel image" \cite{linuxPrimer}[sec. 4.3.1] containing all (required) settings.

The \textit{MicroBlaze} processor can be configured with different feature sets (multiplier, barrel shifter, etc.). Therefore the \gls{gcc} compiler needs to be parameterized for matching the provided features of the target system \cite{mb_linux}[sec. "Kernel Configuration Details"]. These need to be set in the \texttt{XILINX\_MICROBLAZE0\_*} settings inside of the \texttt{.config} file. The \texttt{KERNEL\_BASE\_ADDR} is also an imported setting which need to match the base address of the systems main memory.

The \texttt{xps\_ll\_temac} driver is used for the Ethernet interface.

Another part of the system configuration is the \textit{Device Tree}. It is an abstraction layer for accessing hardware information, to avoid all these into assembler code \cite{device_tree}. It is accessed by the Linux kernel during the boot process for configuring itself and on lookup of hardware information. Therefore the \textit{Device Tree Source (dts)} file is compiled by the \textit{Device Tree Compiler (dtc)} during the Linux kernel build process to an \textit{Device Tree Blob (dtb)} and linked into the final Linux kernel image. It can be generated using the \textit{Device Tree Generator}, a \gls{tcl} script reading a system specification generated by \gls{xps}.

\subsection{The File System}
\label{subsec:fs}

Although possible, it makes little sense to use the Linux kernel without a file system. Therefore one of the last steps in the boot process is the initialization of a \textit{root file system (rootfs)}. \cite{linuxPrimer}[sec. 6.1] On modern desktop and server systems this \textit{rootfs} is usually just a bare minimum file system, containing all necessary files to boot the Linux kernel, mounting a file system located on a hard drive or flash memory right after booting up. But the used system in this project has no hard drive attached, neither is one required for the purpose of the project, at the moment. Therefore we will stick with the initial \textit{rootfs} as the main file system.

A simple way to provide an initial \textit{rootfs} is the \textit{Initial RAM Disk (initrd)}. This is a file system packed in a \textit{cpio} archive and linked into the Linux kernel image. It is unpacked completely into the main memory during kernel boot process. \textit{Xilinx} provides two packed file system archives for a \textit{MicroBlaze} system within their Linux kernel repository: \texttt{initramfs\_minimal.cpio.gz} and \texttt{initramfs\_complete.cpio.gz}. Both contain all files and structures sufficient for this project. The archives can be linked into the Linux kernel image by setting the configuration option \texttt{INITRAMFS\_SOURCE} to one of the file names, respectively.

\textit{cpio} archives can be unpacked using the following command. This command should be executed as privileged root user to allow the creation of node points, used in the file system.

After all changes were made to the files representing the file system, it can be packed into a \textit{cpio} archive using the bash script supplied in the appendix of this report ("\texttt{pack-fs.sh}", \ref{subsec:pack-fs}). The paths to the Linux kernel sources and file system root need to be adjusted to meet the current environment.

\section{Deployment}

The generated \textit{bitstream} of the hardware system representing the \gls{soc}, needs to be programmed into the \gls{fpga} on every power-on. This is required, because the configuration of the \gls{fpga} being set by the \textit{bitstream} is volatile. Programming the \textit{bitstream} is straight forward and can be accomplished using the \textit{Xilinx iMPACT} tool, which is part of the \textit{Xilinx ISE Design Suite}.

When the Linux kernel build is finished, the resulting \textit{\gls{elf}} file can be found at \texttt{arch/microblaze/boot/simpleImage.<dts-name>}. This file can be loaded into the \gls{fpga} using \textit{\gls{xmd}}, which is also part of the \textit{Xilinx ISE Design Suite}.

\chapter{Modifications to the system}

\section{Hardware}

Data and instructions caches were extended to 64 Kilobytes. Because it gives a slight performance gain for the system and the used \gls{fpga} had sufficient resources left.

\section{Software}

The toolchain used during the preceding project was updated to the latest version provided by \textit{Xilinx}. This version has the identifier "\texttt{microblaze-unkwnown-linux-gnu-.... (crosstool-NG 1.14.1) 4.6.2 20111018 (prerelease)}". The \gls{gcc} version included in this version is \texttt{4.6.2}. This \gls{gcc} version is not compatible with the Linux kernel version used previously. Therefore the Linux kernel was updated to version 3.5.0 (commit: \texttt{45b74487f57324fa66da40cd4d52be6f07e2aefd}\footnote{\url{http://git.xilinx.com/?p=linux-xlnx.git;a=commit;h=45b74487f57324fa66da40cd4d52be6f07e2aefd}}). The toolchain upgrade itself was required for the work on user space applications described in the following chapters.

The utilized web server software (nginx) requires the availability of an event module in the \gls{os}. Therefore \texttt{epoll} was activated in the Linux kernel configuration (\texttt{.config} file). This is done by setting the switch \texttt{CONFIG\_EPOLL=y}. Further explanations on this change can be found in the related chapter about the nginx architecture (see \ref{sec:nginx-arch}).

Another event, affecting the software part of this project was the corporate takeover of \textit{PetaLogix} by \textit{Xilinx} in late August 2012 \cite{takeover}. In consequence the efforts of both companies on Linux kernel development for the \textit{MicroBlaze} architecture were bundled, leading -- amongst others -- to a common source code repository for related developments. This alleviated the software part of the project.