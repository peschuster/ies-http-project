\chapter{Conclusion}

\section{Project Review}

I chose this project, because it promised to cover many topics of personal interest, including network systems, performance oriented research and optimization, web technology and System on chip design. In retrospective these expectations were fulfilled, but came with a steep learning curve and many unexpected challenges.

Work on the bachelor thesis covered the second part of a larger project. Therefore a deep knowledge of the general topic was already available and it was possible to work on the project objectives right from the start.

Practical implementations always come with a great deal of uncertainty regarding occurring problems. In this part of the project, these turned out to be initial incompatibilities between compiler, \textit{Linux kernel} source and \textit{nginx} source versions. As well as memory leaks in \textit{nginx}'s request processing, which seemed to stay unresolvable for a major part of the project time.

One thing that turned out very valuable, was the consequent application of automation, where ever possible. This lead to hassle-free implement-build-deploy-cycles facilitating quick changes with great impact also in a late stage of the project.

Original objective of this project and main reason I chose this as a combined project seminar and bachelor thesis, was the integration of a \textit{TCP Offload Engine} into a System on Chip and Linux network stack. When it turned that the \textit{TCP Offload Engine} project, being part of a dissertation, will not the necessary majority level to be integrated into another system in time, this thesis turned into a pure software project. Making the shift from the original objectives to purely preparatory work for a possible integration, was one of the most challenging aspects, from a motivation perspective. 

In review I personally would assess the project as still being successful. Mainly because of all the accomplished integration work up to the reliable running \textit{nginx} instance and all the insights gained on the inner workings of Linux kernel and its network stack implementation.

Due to external circumstances, the original objective of the project was not reached. Therefore an obvious next step would be to finally integrate the \textit{TCP Offload Engine} into a Linux kernel build running on a custom \textit{System on Chip}. Besides this there are always possibilities for optimization of previously implemented features, but without integration of the \textit{TCP Offload Engine}, these probably would not yield a great gain in the overall system performance.