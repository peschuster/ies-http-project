\chapter{nginx}
\label{ch:nginx}

\section{Introduction}

The context of this project is to implement a testbed for a \textit{\gls{toe}} being developed in another research project. This testbed should enable testing the \gls{toe} in an environment and with workloads close to usage in real world. One usage of \gls{tcp} is as protocol layer for \gls{http} requests. \gls{http} is the second most used protocol among all internet traffic, with a share of about 20 to 30 \% (in 2008/09) \cite{internet_study} and therefore very relevant for usage outside lab environments. 

The server-side endpoint for \gls{http} traffic is a web server software. For this project nginx (pronounced "engine-x") was chosen, which follows an asynchronous architecture. It was released to the public in 2004 and focuses on "high performance, high concurrency and high memory usages"\cite{aosa}.

The asynchronous architecture is accomplished by using an event-pattern, instead of thread-based processing. This means - very much simplified - that \textit{nginx} never "waits" during processing a request for any external operation to complete, but pushes it to an event system, does something other useful and picks up the event, when it is finished for further processing steps.

\section{Architecture}
\label{sec:nginx-arch}

\subsection{Overview}

A running nginx instance always consists of atleast two processes: a master and a worker process. The master process spins-up, monitors and controls the worker processes. The worker processes process the actual (HTTP) requests on a single thread.

\subsection{Event-driven Architecture}



\section{Configuration and Building}
\label{sec:nginx-config}

\subsection{Extending the Configuration System}

The build process of nginx can be configured with a number of parameters and constants to add or remove optional modules. The inclusion of modules can be configured with command line arguments to the configuration tool, but many of the parameters can not be set externally.

The values of these parameters are determined by a custom "auto configuration" tool. This tool writes small c programs to a temporary file, compiles them using the configured compiler and reads back the results. By this process nginx adjusts its own build process to the features and properties of the current system. This configuration process is not working for cross-compiling nginx for another target system. Therefore the configuration process needed to be extended to allow input of required configuration parameters into the build process from an external source.

The implemented solution is based on a suggested patch by Daniele Salvatore Albano\footnote{\textit{Cross compilation support for nginx}, Daniele Salvatore Albano, 01/03/2011 \url{http://web.archiveorange.com/archive/v/Tuw7Ryz8rztiNaIFfqCg}}, but incorporates more parameters and is streamlined to the process of cross-compiling the Linux kernel. Cross-compiling can be enabled by setting the environment variable "\texttt{CROSS\_COMPILE}" to the utilized compiler tool chain prefix. For the MicroBlaze tool chain this is "\texttt{microblaze-unknown-linux-gnu-}".

Parameters that are covered by this modification to the configuration system include endianness (\texttt{with-endian}), the size of primitive data types (\texttt{with-int}, \texttt{with-long}, etc.) and the maximum error number, to be used for custom error types (\texttt{with-sys-nerr}).

Value for these introduced parameters can be determined through a small test program executed on the \gls{soc} (see appendix \ref{sec:nginx-env-eval}). This leads to the following values:

\begin{verbatim}
    --with-endian=big \
    --with-sys-nerr=132 \
    --with-int=4 \
    --with-long=4 \
    --with-long-long=8 \
    --with-ptr-size=4 \
    --with-sig-atomic-t=4 \
    --with-size-t=4 \
    --with-off-t=4 \
    --with-time-t=4
\end{verbatim}

\subsection{Modules}

nginx consists of a number of (optional) modules. These modules can be in- or excluded from the build using parameters to the configuration tool. Objective of the nginx build for the MicroBlaze system was a small binary with just the necessary parts included. There for the following modules were excluded:

\begin{verbatim}
    --without-http_rewrite_module \
    --without-http_gzip_module \
    --without-http_charset_module \
    --without-http_ssi_module \
    --without-http_userid_module \
    --without-http_access_module \
    --without-http_auth_basic_module \
    --without-http_autoindex_module \
    --without-http_status_module \
    --without-http_geo_module \
    --without-http_map_module \
    --without-http_split_clients_module \
    --without-http_referer_module \
    --without-http_proxy_module \
    --without-http_fastcgi_module \
    --without-http_uwsgi_module \
    --without-http_scgi_module \
    --without-http_memcached_module \
    --without-http_limit_conn_module \
    --without-http_limit_req_module \
    --without-http_empty_gif_module \
    --without-http_browser_module \
    --without-http_upstream_ip_hash_module \
    --without-http_upstream_least_conn_module \
    --without-http_upstream_keepalive_module \
    --without-pcre \
    --without-select_module \
    --without-poll_module
\end{verbatim}

Some of these modules could be included in the build, if necessary, but the modules \texttt{http\_rewrite\_module} and \texttt{http\_gzip\_module} require external libraries which are not present on the MicroBlaze system and therefore would not work.

\subsection{Compiler Configuration}

Besides the configuration for nginx itself, the compiler needs to be configured for the target MicroBlaze system. When building the Linux kernel, the compiler configuration is set by the configuration of the kernel during the build process. nginx does not have this functionality in its configuration system, but comes with a parameter (\texttt{--with-cc-opt=...}) to pass custom parameters to the compiler.

Through this configuration setting, the gcc cross-compiler needs to be configured for the feature set of the MicroBlaze system. For the presented MicroBlaze \gls{soc}, the required parameters are

\begin{verbatim}
-mxl-multiply-high -mno-xl-soft-mul -mno-xl-soft-div \
-mxl-barrel-shift -mxl-pattern-compare -mcpu=v8.30.a
\end{verbatim}

Additionally the path to the standard libraries on the system needs to be set through the \texttt{--sysroot} parameter. These libraries are part of the tool chain for \textit{MicroBlaze} systems provided by \textit{Xilinx}. 

By specifying the \texttt{--static} parameter all referenced libraries are build into the resulting binary. Therefore the binary has less dependencies to be executed.

Combining it all together the value of the \texttt{--with-cc-opt} parameter needs to be set to something like the following: 

\begin{verbatim}
--with-cc-opt="-mxl-multiply-high -mno-xl-soft-mul -mno-xl-soft-div 
-mxl-barrel-shift -mxl-pattern-compare -mcpu=v8.30.a --static 
--sysroot=/home/peschuster/project/microblaze-unknown-linux-gnu/microblaze-unknown-linux-gnu/sys-root"
\end{verbatim}

\subsection{Memory Leaks}

One problem that arose during tests were memory leaks. It turned out that nginx allocated about 4053.2 Bytes of memory for each request. Assuming available memory of about 200 MB, nginx crashed the complete system after approximately 50,500 requests in total. of course, this is an unacceptable behavior for a (web) server system.

\begin{figure}[H]
\begin{minipage}{0.4\textwidth}
\begin{tabular}{|r|r|r|}
    \hline
     \textbf{requests} & \textbf{master / KB} & \textbf{worker / KB} \\
    \hline
    0     & 3064  & 3204 \\
    1000  & 3064  & 7296 \\
    2000  & 3064  & 11256 \\
    3000  & 3064  & 15348 \\
    4000  & 3064  & 19440 \\
    5000  & 3064  & 23404 \\
    6000  & 3064  & 27496 \\
    7000  & 3064  & 31588 \\
    8000  & 3064  & 35552 \\
    9000  & 3064  & 39644 \\
    10000 & 3064  & 43736 \\
    \hline
    \end{tabular}
\end{minipage}
\begin{minipage}{0.65\textwidth}
	\centering
	\begin{tikzpicture}
		\begin{axis}[width=\textwidth,height=8cm,
			xlabel={requests},
			ylabel={memory / KB},
			xmin=0,
			ymin=0,
			extra y ticks={3064},
			%extra y tick label style={/pgf/number format/1000 sep=},
			extra y tick style={grid=major},
			extra x tick style={grid=major},
			y tick label style={/pgf/number format/1000 sep=},
			legend pos = north west]
			\addplot table[x index=0, y index=1] {graphdata/nginx-mem.csv};
			\addlegendentry{master process}
			\addplot table[x index=0, y index=2] {graphdata/nginx-mem.csv};
			\addlegendentry{worker process}
		\end{axis}
	\end{tikzpicture}
\end{minipage}
  \caption{Memory consumption of nginx processes over total requests.}
  \label{fig:nginx-mem}
\end{figure}

\subsubsection{Investigations}

The described behavior of nginx could not be replicated on a standard x86 server system running \textit{Ubuntu Linux 12.04}. That means nginx in general should work correct, using just as much memory as required and releasing unused memory to the \gls{os}. However, on the \textit{MicroBlaze} system this does not work properly.

With activated \texttt{debug} log. nginx writes some information about inner workings to the error log. The debug log can be activated with the following option in the configuration file:

\begin{verbatim}
error_log  logs/error.log  debug;
\end{verbatim}

There must be only one \texttt{error\_log} line in the nginx configuration file, but the option specifying the severity level is inclusive for all lower levels.

The following table shows all memory related debug messages as found in the error log for a single request to a static file on the \textit{MicroBlaze} system running a nginx instance, which is configured as described in the previous section (\ref{sec:nginx-config}):

\begin{table}[H]
\centering
\begin{tabular}{|l|l|r|r|}
    \hline
     \textbf{Log entry} & \textbf{ptr} & \textbf{allocated} & \textbf{freed} \\
    \hline \hline
\texttt{*1 malloc: 10080890:644} & 10080890 & 644 &  \\ \hline
\texttt{*1 malloc: 10080B18:1024} & 10080B18 & 1024 &  \\ \hline
\texttt{*1 posix\_memalign: 1007B5B0:4096 @16} & 1007B5B0 & 4096 &  \\ \hline \hline
\texttt{*1 free: 1007B5B0, unused: 2079} & 1007B5B0 & & 4096 \\ \hline
\texttt{*1 free: 10080890} & 10080890 & & 644 \\ \hline
\texttt{*1 free: 10080B18} & 10080B18 & & 1024 \\ \hline
\end{tabular}
\caption{Memory management related debug messages of a single request.}
\label{tab:debug_mem}
\end{table}

All pointers to allocated memory for one request are passed to the \texttt{free(..)} function of the \textit{C Standard Library} (\textit{libc}) and therefore should be released to the system. But performance and stress tests on the system proved a different behavior: The nginx worker process consumes more memory for each request with a linear relation to the number of requests (see \ref{fig:nginx-mem}).

Therefore the error causing this misbehavior probably is not located in nginx itself, but in the underlying system layers. This would mean that memory management of the Linux kernel or \textit{libc} port to the \textit{MicroBlaze} architecture are broken, in the described respect. This is a strong allegation, especially because fully analyzing the problem on this wide dimensions was beyond the scope of this bachelor thesis. But there are a view points promoting this theory which should be considered:

There is an answer by \textit{Xilinx} support (\textit{AR \#12421}) stating that memory management (especially the function \textit{free(..)}) is "very system-specific" and not fully implemented/supported for \textit{MicroBlaze} processors \cite{mbfree}. The answer was published on 09/09/2010 and is explicitly only valid for versions of the \textit{MicroBlaze} processors without a hardware memory management unit. Therefore it should not apply to the used version of the \textit{MicroBlaze} processor and \textit{C Standard Library}, but it shows that these memory management functions were added to the toolchain and supporting environment only recently and might not be as stable as other parts.

Another point benefiting this theory is the unusual way nginx deals with memory. Memory management is done by nginx's pool allocator. This could be seen as an abstraction to the memory allocation mechanisms provided by the \gls{os}. When a module/sub-routine/function requires dynamically allocated memory, it requests it from the nginx pool allocator, which itself requests a larger chunk of memory (called "pool") from the \gls{os} and distributes small blocks of the pool on requests from other parts of nginx. When no blocks of a pool are used anymore, it is returned to the \gls{os}. nginx uses this design to minimize system calls and reduce expensive requests for memory allocation by the hardware memory management unit \cite{aosa}. Another consequence of this design is that nginx does not reuse once allocated memory, but just allocates new memory blocks when required and releasing them to the system upon finished operations. This is by design and beneficial for speed and efficiency, but comes in unfavorable, when the memory management of the system (\gls{os} and processor) might not work properly.

\subsubsection{A First Workaround}

It was not possible to fix the root cause of the memory leaks during this bachelor thesis. To circumvent the described arising problems, another solution needed to be found to be able to use the system and conduct extensive performance tests.

A workaround to circumvent complete system crashes during tests due to exhausted memory is to restart the nginx worker process on low remaining system memory.

This can be accomplished by sending the \texttt{HUB} signal to the nginx master process using the \texttt{kill} command of \textit{Unix} systems\footnote{\url{http://unixhelp.ed.ac.uk/CGI/man-cgi?kill}}. The \texttt{HUB} signal tells nginx to reload the current configuration, resulting in spinning up new worker processes and gracefully shutting down the previous ones. This differs from complete restarts of nginx in the way that no incoming requests are lost during the restart process\footnote{\url{http://wiki.nginx.org/CommandLine} (as of 12/2012)}.

The process id of the master process on the system under test is always stored in the file \texttt{/usr/local/nginx/logs/nginx.pid}. Therefore the complete command for restarting the nginx worker process can be constructed as follows:

\begin{verbatim}
kill -HUP $( cat /usr/local/nginx/logs/nginx.pid )
\end{verbatim}

Information about currently allocated and free memory can be displayed using tools like \textit{top}\footnote{\url{http://www.busybox.net/downloads/BusyBox.html\#top}}, which provide "a view of process activity in real time" \cite{busybox}. \textit{top} internally aggregates information from multiple (virtual) file handles like e.g. \texttt{/proc/meminfo} for information about memory.

\subsubsection{An Integrated Solution}

This workaround works well, but requires manual actions by a user and is therefore ignorant for automation which is a key part of extensive performance tests.

An idea for an extended solution was to shift checking for low remaining memory to the nginx master process.

After configuring and building up the system, including the start of worker processes, the master process returns to a standby mode, waiting for external signals (like the previously described \texttt{HUB} signal). This is implemented using the default \textit{C Standard Library} signal implementation, mainly \texttt{sigsuspend}\footnote{\url{http://www.gnu.org/software/libc/manual/html\_node/Sigsuspend.html}}, inside of the \texttt{ngx\_master\_process\_cycle(..)} function (in the file \texttt{ngx\_worker\_cycle.c}).

A developed patch to check memory consumption of the system and initiate worker process restarts hooks in at this point inside of the master process. 

The patch consists of three major parts:

\begin{enumerate}

\item Starting a timer, to awake the master process every second from suspension. This is implemented using the Linux command \texttt{setitimer}\footnote{\url{http://linux.about.com/library/cmd/blcmdl2\_setitimer.htm}} in \texttt{ITIMER\_REAL} mode to send a \texttt{SIGALRM} signal to the hosting (i.e. nginx master) process.

\item A signal handler for dealing with the \texttt{SIGALRM} signal, to check the remaining system memory and trigger a restart of the worker processes, if necessary.

\item A function reading the file handle \texttt{/proc/meminfo}, parsing out the required information about free system memory and comparing it with a configured threshold value. This is implemented in the new file \texttt{ngx\_process\_memguard.c}.
\end{enumerate}

Usage of the newly implemented functionality needs to be activated upon build configuration by providing the parameter \texttt{----with-min-free-mem=[value]}. \texttt{[value]} is the threshold value in bytes which is compared to the free system memory to decide on a required restart of worker processes. During tests of the system this value was set to "\texttt{51200}" (kilobytes).

The complete patch is included in the appendix of this report (see \ref{appendix:memguard}) or can be found in the master branch of the nginx fork at \url{https://github.com/peschuster/nginx}.

\section{Interfaces to the Operating System}

This section briefly describes interactions between nginx and the underlying \gls{os} to deal with HTTP requests and responses.

socket

accept

Responses for requests to static files are written to the socket buffer using the two functions \texttt{writev} (for HTTP header data) and \texttt{sendfile} for content of the actual static file. This leads to atleast two TCP packets, always (see introduction to performance tests: \ref{subsec:mtu}).

recv